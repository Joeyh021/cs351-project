\chapter{Project Management}

In this chapter we describe the methodology and tools used to undertake the project.

\section{Methodology}
The project is very practical, with much of the time spent writing and testing HDL code. As such, an appropriate engineering methodology is employed.

Our high-level approach is very plan based: we start with the design outlined in Chapter \ref{ch:design}, detailing the instruction set and then block design of the PIO device. Having a detailed design on paper before any code is written has it's advantages: the design is well documented, and allows to focus on the implementation details of each module while writing code instead of worrying about higher level architectural details.

Implementation begins with taking the block design in Chapter \ref{ch:design}, identifying relevant HDL modules, and building a skeleton Scala project with all the Chisel modules required, as detailed in Chapter \ref{ch:implementation}. Once this structure is in place, we implement the modules one at a time, testing each as we go. Any iteration on the high level block design required going back

On a high level this is a very structured methodology. waterfall, etc. Good for a complex design and makes easy to document

However, on a module level very agile, extreme programming, test driven development, etc. Easy with a small development team with no external stakeholders.

\section{Schedule}
Project started, original timeline in appendix
actual timeline given in table 
due to xyz restrictions
tasks and timing used kaban board 
\section{Source Control}
git 
github 
CI


\section{Tools \& Resources}
git&github
scalac
sbt
chisel libraries
vivado
ubuntu workstation
fpga board