\chapter{Project Management}

In this chapter we describe the methodology and tools used to undertake the project.

\section{Methodology}
The project is very practical, with much of the time spent writing and testing HDL code. As such, an appropriate engineering methodology is employed. Figure \ref{fig:methodology} gives an overview of our development process and design flow.

Our high-level approach is very plan based: we start with the design outlined in Chapter \ref{ch:design}, detailing the instruction set and then block design of the PIO device, defining all the HDL modules needed and the relationships and connections between them. Having a detailed design on paper before any code is written has it's advantages: the design is well documented, and allows to focus on the implementation details of each module while writing code instead of worrying about higher level architectural details.

Implementation begins with taking the block design in Chapter \ref{ch:design}, and building it out into a skeleton Scala project with all the Chisel modules required as detailed in Chapter \ref{ch:implementation}. Once this structure is in place, we implement the modules one at a time, writing and testing code in an iterative fashion. Any issues identified in the design during implementation were first fixed in the block design or instruction set, and then went back through the entire design process from there, fixing any issues as we followed the design flow through.

On a high level this is a very structured, plan-based methodology. It borrows from the waterfall methodology: first defining requirements and objectives, laying out a detailed design, and then implementing based on the design. Going back to the start and following back through the design process when any changes need to be made is another trait borrowed from the waterfall methodology \cite{softeng}. This kind of methodology is ideal for a project such as ours that is both complex and needs to be well documented.

Where this methodology differs from waterfall is when it comes to the implementation on a per-module level. The opposite approach is taken, leaning into a much more iterative, agile style of development, borrowing ideas of incremental development and constant testing from extreme programming methodologies. This approach is very natural for a development team of only one person, and easy when there are no external stakeholders or customers \cite{softeng}.

The testing process is also inspired by agile methodologies. Test-driven development is used heavily, writing unit tests for each module alongside the module itself, with the implementation and functional verification processes interleaved. Chisel's close integration with it's testing and verification tools (ChiselTest \& Treadle, as mentioned in Chapter \ref{ch:evaluation}) makes it easy to rapidly write testbenches. The codebase includes a large test suite as a cause of this, which allows to develop and iterate quickly and with confidence.

End-to-end system-level verification also takes place: verifying that the design synthesises correctly, and testing the system with the test program as outlined in Chapter \ref{ch:evaluation}. This system validation process rounds off the tail end of the waterfall methodology we employ.


\begin{figure}[H]
    \centering
    \includegraphics[width=0.6\textwidth]{../img/methodology.png}
    \caption{An overview of our development methodology}
    \label{fig:methodology}
\end{figure}


\section{Timeline}
The project commenced at the start of October 2022 (Week 1) with the submission of the project specification. The timeline as stated in the specification (Timeline A) is given in Table \ref{tab:timeline}, along with the timeline as revised at the end of November 2022 (Week 9) (Timeline B), and the actual timeline which the project followed (Timeline C).

\begin{table}[ht!]
    \centering
    \begin{tabular}{|p{0.64\linewidth}|p{0.12\linewidth}|p{0.12\linewidth}|p{0.12\linewidth}|}
        \hline
        \multirow{2}{*}{\textbf{Task}}                                      & \multicolumn{3}{c|}{\textbf{Week(s)}}                                             \\ \cline{2-4}
                                                                            & \textbf{Timeline A}                   & \textbf{Timeline B} & \textbf{Timeline C} \\ \hline
        Background research                                                 & 1-2                                   & 1-4                 & 1-4                 \\ \hline
        Implement proof of concept I/O device in Chisel                     & 3-5                                   & 4-11                & 4-11                \\ \hline
        Lay out high-level block design                                     & 6-8                                   & 12                  & 12-13               \\ \hline
        Extend proof of concept into project skeleton based on block design & 8-11                                  & 13                  & 14                  \\ \hline
        Write HDL code, implementing and testing one module at a time       & 12-18                                 & 14-19               & 15-19               \\ \hline
        Develop and run integration tests in simulation                     & 19                                    & 20                  & 20-23               \\ \hline
        Use Rocket Chip to integrate device with RISC-V cores               & 20-21                                 & 21-22               & 21                  \\ \hline
        Load microcontroller onto FPGA and run tests                        & 22-24                                 & 23-24               & 22-23               \\ \hline
        Write report                                                        & 25-31                                 & 25-31               & 24-31               \\ \hline
    \end{tabular}
    \caption{Project Timeline}
    \label{tab:timeline}
\end{table}

Very little project work took place in the first 10 weeks, due to being busier with other modules and coursework. More time was also spent getting to grips with Chisel and Scala than anticipated, figuring out effective workflows for integrating Chisel and it's compilation process into Vivado's design flow while developing a very basic proof-of-concept. However, this time was not wasted, as it gave a strong foundation on which to build during development.

After week 13 when the block design was complete, the original timeline was mostly caught up to. Development was completed by week 19, when effort shifted to system verification both in simulation and with software. The Rocket chip integration process, simulation and system verification processes also ended up becoming more interleaved towards the end of the project, as lots of time was spent debugging and iterating back through the design process. The timeline was cut short by a week due to the need to present results as part of the project presentation at the end of week 23, meaning the test software only demonstrated very basic capabilities.

The timeline and associated tasks were tracked using an interactive Kanban board \footnote{\url{https://github.com/users/Joeyh021/projects/2/}} as part of GitHub's project features. Each of the tasks listed in Table \ref{tab:timeline}, as well as any other tasks that came up, were added to the board and categorised as either `Todo', `In Progress', `Completed' or `Blocked'. Comments were left on each task to keep notes on progress, and reference specific Git commits or code associated with the tasks. This was very valuable for keeping track of progress, especially as work ramped up and multiple tasks were being worked on towards the end of the project.

\section{Source Control}
git
github
CI


\section{Tools \& Resources}
git\&github
scalac
sbt
chisel libraries
vivado
ubuntu workstation
fpga board