\begin{abstract}
    Traditionally, general-purpose embedded systems necessitate the inclusion of multiple hardware interfaces such as I2C, SPI, and UART, to allow interfacing with a variety of devices. General Purpose I/O (GPIO) pins are also often included for more general interfaces but cannot be used with high-speed peripherals with precise timing requirements.

    We explore Programmable I/O as an alternative: flexible and highly performant I/O devices that allow for communication at high-speed using whatever protocol a programmer wishes into implement. We then integrate PIO into open-source RISC-V cores to creates a flexible and compact MCU that can be used in low power, low cost embedded applications where adaptable or custom interfaces are required, without the need for custom hardware.

    Through this we also evalute Chisel, a novel hardware description language, and find it to be a compelling alternative to Verilog.

    \textbf{Keywords:} PIO (Programmable I/O) , Chisel
\end{abstract}