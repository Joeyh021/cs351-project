\begin{abstract}
    Traditionally, general-purpose embedded systems necessitate the inclusion of multiple hardware interfaces such as I2C, SPI, and UART, to allow interfacing with a variety of devices. Microcontrollers include a wide array of fixed interfaces, yet it is rare that a device's I/O complement meets exactly the needs of an engineer.

    We explore Programmable I/O (PIO) as an alternative: flexible and highly performant I/O devices that allow for communication at high-speed using whatever protocol a programmer wishes into implement. We then integrate PIO with open-source RISC-V cores to creates a flexible and compact MCU that can be used in applications where adaptable or custom interfaces are required without the need for custom hardware.

    We find PIO to occupy similar power and hardware resources to a standard hardware interface such as SPI, while also being able to implement SPI and other standards, as well as non-standard communication protocols such as WS2812B serial.

    Through this we also evalute Chisel, a novel hardware description language, and find it to be a compelling alternative to Verilog.

    \textbf{Keywords:} PIO (Programmable I/O) , Chisel
\end{abstract}