\begin{abstract}
    Traditionally, general-purpose embedded systems necessitate the inclusion of multiple hardware interfaces such as I\textsuperscript{2}C, SPI, and UART, to allow interfacing with a variety of devices. Microcontrollers include a wide array of fixed interfaces, yet it is rare that a device's I/O complement meets exactly the needs of an engineer.

    This work explores Programmable I/O (PIO) as an alternative: flexible and highly performant I/O devices that allow for communication at high speed using whatever protocol a programmer wishes to implement. We then integrate PIO with open-source RISC-V cores to create a flexible and compact MCU that can be used in applications where adaptable or custom interfaces are required without the need for custom hardware.

    We find our PIO implementation to be capable of implementing common standard protocols such as SPI, as well as non-standard interfaces such as WS2812 serial. When implemented in a Xilinx Artix 7 device, up to an order of magnitude more FPGA resources are utilised by PIO than sample full-featured single purpose I/O interface hardware. Despite its high utilisation relative to fixed interfaces, it is still considered small when included as part of a larger SoC design, and has the advantage of being far more versatile.

    Through this we also evaluate Chisel, a novel hardware description language, and find it to be a compelling alternative to Verilog.

    \textbf{Keywords:} PIO (Programmable I/O), Chisel, FPGA, RISC-V
\end{abstract}