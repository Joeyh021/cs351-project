\chapter{Introduction}
\label{ch:introduction}

An embedded system is, loosely defined, `any device that includes a programmable computer but is not itself intended to be a general-purpose
computer'~\cite{wolf}. A laptop or desktop PC is not an embedded system, but digital clocks, calculators, televisions, and domestic appliances such as refrigerators and dishwashers are all examples of embedded systems.

Embedded systems are typically built around microcontrollers, small CPUs that are designed to provide computing functions within the context of a larger electromechanical system. As such, embedded system design usually starts with interface and I/O requirements. Microcontroller units (MCUs) have to be able to interface with the world around them via sensors, actuators, LEDs, etc, but communicating with external hardware can be difficult due to the need for high frequencies and precision timing.

General-purpose interfaces such as UART, SPI, and I\textsuperscript{2}C are commonly found in microcontrollers and facilitate communication with a broad range of common electronic devices via standard protocols. However, different interfaces have different requirements which necessitates the inclusion of dedicated fixed-function hardware for each communication protocol a system wishes to support. Each hardware interface adds cost and complexity to development, as well as the cost in system power consumption and chip area. General-purpose microcontrollers typically include a variety of hardware to be as flexible as possible, which almost always results in unnecessary expense as it is rare that a MCU's full complement of I/O hardware is in use for any given use case. It is also sometimes the case that there is not enough of a single type of interface within a system, i.e., you have 2 SPI and 2 UART ports but you need 4 UARTs, which can leave designers in a difficult position.

The solution we explore is programmable I/O, or PIO. First introduced by Raspberry Pi with the launch of their RP2040 MCU in 2021, PIO is a hardware interface that is programmable in the same sense as a CPU via a small assembly language (pioasm). Instructions are executed sequentially by the PIO to control signalling and transfer data between the PIO block, external hardware and the rest of the system. PIO is highly performant, able to control signalling at a much higher speed than the CPU due to it's specialised nature. Highly configurable system behaviour and flexible mapping of inputs and outputs to GPIO pins make PIO flexible enough to implement most common I/O protocols, as well as any other custom or uncommon standards~\cite{rp2040}.

This work implements a PIO device that is based on that of the RP2040, and is integrated with open-source RISC-V cores to create a proof-of-concept low-power, low-cost, flexible microcontroller that can be used in embedded applications which may have very specific or custom I/O requirements which cannot be met by other off-the-shelf systems, in addition to all the typical use-cases. The PIO device is also be open sourced to provide an open source implementation of the PIO assembly language, and to contribute back to the open source hardware community.

The RISC-V implementation we use is Rocket chip, an open-source SoC generator framework available on GitHub under the BSD license~\footnote{\url{https://github.com/chipsalliance/rocket-chip}}. Rocket chip is extensively customisable, partly through it's use of Chisel as an implementation language.

Chisel is a hardware construction language that supports hardware design using parametrised generators and a domain specific hardware language. It is embedded within Scala, a hybrid functional/object-oriented JVM language, which allows it to support higher levels of abstraction than traditional hardware languages such as Verilog and VHDL. It's more modern features and embedding within Scala position it as an interesting alternative to Verilog, and through the implementation of PIO, we evaluate it's efficacy as a language for designing hardware on the register transfer level (RTL).\cite{chisel}.